%% =====================================================================
%% FONT CONFIGURATION
%% =====================================================================
%% 'fontspec' is a LaTeX package that provides an interface for loading
%% OpenType and TrueType fonts by name when using XeLaTeX or LuaLaTeX.
%% Without fontspec, LaTeX can only use its own built-in font system.
\usepackage{fontspec}

%% \setmainfont sets the primary text font for the entire document.
%% Because we are using XeLaTeX + fontspec, we can reference any font
%% installed on the system by its name. Times New Roman is located at:
%%   /usr/share/fonts/truetype/msttcorefonts/
%% fontspec finds it automatically by name, so no path is needed.
%%
%% This replaces the previous EB Garamond configuration, which required
%% explicit Path, Extension, and font-file mappings because those fonts
%% were installed in a non-standard location.
\setmainfont{Times New Roman}

%% =====================================================================
%% LIST FORMATTING
%% =====================================================================
%% 'enumitem' provides fine-grained control over list environments
%% (itemize, enumerate, description). We use it below to make bullet
%% lists compact (no extra spacing) for a dense CV layout.
\usepackage{enumitem}

%% =====================================================================
%% PAGE STYLE AND SPACING
%% =====================================================================
%% \pagestyle{empty} removes headers, footers, and page numbers.
%% This is standard for CVs — page numbers add clutter and the document
%% length is obvious from context.
\pagestyle{empty}

%% \linespread{0.95} tightens line spacing to 95% of the default.
%% \selectfont activates the change immediately. This gives the CV a
%% compact, professional appearance without feeling cramped.
%% Override the base font size to 8pt. The second argument (9.6pt)
%% is the baseline skip (line spacing), set to 1.2x the font size.
%% \selectfont activates both this and the \linespread together.
\fontsize{8pt}{9.6pt}
\linespread{0.95}\selectfont

%% Remove all paragraph spacing and indentation. In a CV, content is
%% structured by sections and commands, not by paragraph flow, so
%% default paragraph formatting would just waste space.
\setlength{\parskip}{0pt}      %% No extra space between paragraphs
\setlength{\parindent}{0pt}    %% No first-line indentation

%% Configure all itemize (bullet) lists to be compact:
%%   leftmargin=*   → auto-calculate minimal left margin
%%   topsep=0pt     → no space above/below the list
%%   itemsep=0pt    → no space between items
%%   parsep=0pt     → no space between paragraphs within an item
\setlist[itemize]{leftmargin=*,topsep=0pt,itemsep=0pt,parsep=0pt}

%% =====================================================================
%% LAYOUT DIMENSIONS — TWO-COLUMN CV GRID
%% =====================================================================
%% The CV uses a two-column layout: a narrow left column for section
%% labels (e.g., "EDUCATION", "WORK EXPERIENCE") and a wide right
%% column for the actual content. These three lengths define that grid.

%% Left column width: holds the section label (e.g., "EDUCATION").
%% 3.0cm is wide enough for most labels while leaving plenty of room
%% for content on the right.
\newlength{\lcolwidth}
\setlength{\lcolwidth}{3.0cm}

%% Gap between the left label column and the right content column.
\newlength{\colgap}
\setlength{\colgap}{0.3cm}

%% Right column width: computed automatically as whatever space remains
%% after subtracting the left column and the gap from the full text
%% width. \dimexpr performs arithmetic on LaTeX dimensions.
\newlength{\rcolwidth}
\setlength{\rcolwidth}{\dimexpr\textwidth-\lcolwidth-\colgap\relax}

%% Width of the right-side "date box" inside entries. This box holds
%% the date range and location, centered as a unit. 5.2cm comfortably
%% fits dates like "January 2022 - May 2022" with breathing room.
\newlength{\dateboxwidth}
\setlength{\dateboxwidth}{4.6cm}

%% =====================================================================
%% CUSTOM COMMANDS
%% =====================================================================
%% These commands are the building blocks of every CV section. They
%% enforce consistent formatting so you never have to manually align
%% columns or remember spacing values.

%% ----- \cvrow{LABEL}{CONTENT} -----
%% The fundamental two-column row. Places a bold label in the left
%% column and arbitrary content in the right column.
%%   #1 = Section label text (e.g., "EDUCATION", "WORK\\EXPERIENCE")
%%        Use \\ inside the label for line breaks in multi-word labels.
%%   #2 = Right-column content (can be anything: text, lists, tables)
%%
%% \parbox[t] creates a box with top-aligned text. This ensures the
%% label and content align at the top even when the right column has
%% many lines. \centering centers multi-line labels (like "SELECTED
%% PUBLICATIONS") horizontally within the narrow left column.
\newcommand{\cvrow}[2]{%
  \noindent\parbox[t]{\lcolwidth}{\raggedright\textbf{#1}}%
  \hspace{\colgap}%
  \parbox[t]{\rcolwidth}{#2}%
}

%% ----- \cvrule -----
%% A full-width horizontal rule to separate CV sections visually.
%% 0.4pt is the line thickness; 4pt of vertical space is added above
%% and below the rule for breathing room.
\newcommand{\cvrule}{\vspace{4pt}\noindent\rule{\textwidth}{0.4pt}\vspace{4pt}}

%% ----- \cventry{TITLE}{ORG}{DATE}{INSTITUTION}{LOCATION}{BULLETS} -----
%% A structured entry for work experience, teaching, etc. Produces:
%%   **Title** | Org                          Date
%%   Institution                           Location
%%   - bullet 1
%%   - bullet 2
%%
%% The entry uses two side-by-side \parbox regions:
%%   Left box  — title/org on line 1, institution on line 2
%%   Right box — date on line 1, location on line 2, both centered
%% The right box has a fixed width (\dateboxwidth) so dates and
%% locations align as their own visual unit, centered relative to
%% each other with breathing room from the title text.
%%
%%   #1 = Role/position title (rendered bold)
%%   #2 = Organization or group name (after a pipe separator)
%%   #3 = Date range (centered in right box)
%%   #4 = Parent institution (e.g., "Georgia Institute of Technology")
%%   #5 = Location (e.g., "Atlanta, GA")
%%   #6 = Bullet points — must be \item entries (wrapped in itemize)
\newcommand{\cventry}[6]{%
  \noindent
  \parbox[t]{\dimexpr\rcolwidth-\dateboxwidth-0.15cm\relax}{%
    \raggedright\textbf{#1} | #2\\#4%
  }%
  \hfill
  \parbox[t]{\dateboxwidth}{%
    \raggedleft\begin{tabular}[t]{@{}r@{}}#3\\#5\end{tabular}%
  }%
  \vspace{2pt}%
  \begin{itemize}#6\end{itemize}%
}

%% ----- \cventryplain{TITLE}{DATE}{INSTITUTION}{LOCATION}{BULLETS} -----
%% Same as \cventry but WITHOUT the "| Org" separator. Use this when
%% the role title stands alone (e.g., "Co-Founder and President of
%% Puerto Rican Student Association" already includes the org name).
%% Same two-box layout as \cventry — left for title/institution,
%% right for centered date/location.
%%
%%   #1 = Role/position title (rendered bold)
%%   #2 = Date range (centered in right box)
%%   #3 = Parent institution
%%   #4 = Location
%%   #5 = Bullet points — must be \item entries
\newcommand{\cventryplain}[5]{%
  \noindent
  \parbox[t]{\dimexpr\rcolwidth-\dateboxwidth-0.15cm\relax}{%
    \raggedright\textbf{#1}\\#3%
  }%
  \hfill
  \parbox[t]{\dateboxwidth}{%
    \raggedleft\begin{tabular}[t]{@{}r@{}}#2\\#4\end{tabular}%
  }%
  \vspace{2pt}%
  \begin{itemize}#5\end{itemize}%
}

%% ----- \cvaward{TITLE}{DATE}{DESCRIPTION} -----
%% A compact entry for honors and awards. Produces:
%%   **Title**                                              Date
%%   Description text
%%
%%   #1 = Award name (rendered bold)
%%   #2 = Date received (right-aligned)
%%   #3 = Short description or context
%% \\[1pt] adds a tiny 1pt vertical gap between the title line and
%% the description. \par ends the paragraph and  adds
%% space before the next award entry.
\newcommand{\cvaward}[3]{%
  \textbf{#1} \hfill #2\\[1pt]
  #3\par\vspace{8pt}%
}
